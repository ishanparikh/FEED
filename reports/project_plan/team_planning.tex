\section{Time Management}
    \subsection{Team Organisation}
        For our development process, we decided to split the project into 3 main areas. These cover our main development areas; robotics, application and server. Our robotics team consists of 4 people. Our application and server team consist of 2 people each. The breakdown is as follows: \\

        \subsubsection{Robotics}
            \begin{itemize}
                \setlength{\itemindent}{.2in}
                \item Syakirah
                \item Jerry
                \item Ishan
                \item Brodie
            \end{itemize}

        \subsubsection{Application}
            \begin{itemize}
                \setlength{\itemindent}{.2in}
                \item Cuijing
                \item Asmita
            \end{itemize}
        \subsubsection{Server}
            \begin{itemize}
                \setlength{\itemindent}{.2in}
                \item Nyal
                \item Luc (Project Manager)
            \end{itemize}
            However, both within and across these teams, there will be an overlap in responsibilities. This is to ensure that everyone has a holistic understanding of how each part of the robot functions. The project manager will coordinate across the sub teams to ensure that all the tasks are proceeding smoothly.

    
    \subsection{Workflow Management}
    Workflow management is an essential part of any project. A key part of this is ensuring that everything is communicated clearly and everyone is aware of what needs to be done. 
    
    \subsection{Meetings}
    We will meet as a team with our mentor every week, to update them of our progress and get feedback on how we are doing. Outside of this, we will aim to meet collectively as a team at least one other time each week. This will ensure that everyone is on the same page and we can collaborate our ideas on the project and explain any issues or the specifics of how parts of our system function. To improve efficiency we will do our best to ensure that our meetings are not overly long. Furthermore, we will take minutes of each meeting, this will help unnecessary repetition of ideas across meetings. It will also provide us to develop a plan of what we still need to achieve over the coming weeks. \par
Within our sub teams, we will regularly meetup to discuss what we are working on, as well as to build and construct the system. To ensure that the project is cohesive and everyone is working well, the project manager will coordinate and liaise between the sub teams to ensure that everything is running smoothly.

    \subsection{Communication}
    Our primary method is communication is through Slack. Within this, we have set up different channels for each area of development. This will facilitate discussion through necessary communication channels and make sure that all communications relating to specific areas are all in the same place and easy to find.

    \subsection{Codesharing}
    We will use Github as our method of sharing code. We decided to have 1 repository split into 2 main branches. One for the creation of our app and server, and another for our robotics code. Within these branches we will have  By having 2 separate branches, we are hoping to minimise the necessity to re-pull the entire repository each time a change is made to an area that is not part of the relevant project.

    \subsection{Task Management \& Progress Tracking}
    As previously mentioned, we have split our project into 3 main sub-teams. We are currently using Trello for our main way of managing tasks. We have set up columns for each sub-team that will have the main tasks that are needed to be completed. We have also set up columns to keep track of the deadlines and events over the semester. This is synced with a Google Calendar on Slack, to ensure that everyone is aware of any upcoming deadlines in the week. For specific task management, we have created a project on github, this will allows us to delegate specific tasks to team members, this should help track the finer details of the project. 
    
    \subsection{Document Management}
    
    We are currently using Google Drive for sharing documents. This means that everyone is able access documents easily and we can collaborate when writing report drafts before we put them into \LaTeX.
   
    \subsection{Report Writing}
    When writing reports, we will agree on the rough outline and content together. We will then assign who is responsible for doing each individual section. Ideally, we will have at least two people on each section. This means that there should be sufficient overlap so that every section will have the necessary coverage and detail. Once the sections have been completed, we will peer review each section to make sure that we are all happy with the content. The project manager will then review the report, checking that the overall structure of the report is sound and that it flows naturally. This will also be a check for any spelling or grammar inconsistencies.



